\section*{Phần 7.2}
\subsection*{Bài 7}
Xác suất của các biến cố sau là bao nhiêu khi ta chọn ngẫu nhiên một hoán vị của \{1,2,3,4\}?
\begin{enumerate}[label=\alph*)]
    \item 1 đứng trước 4.
    \item 4 đứng trước 1.
    \item 4 đứng trước 1 và 4 đứng trước 2.
    \item 4 đứng trước 1, 4 đứng trước 2 và 4 đứng trước 3.
    \item 4 đứng trước 3 và 2 đứng trước 1.
\end{enumerate}
\begin{proof}
    Ta có tổng cộng $4!=24$ hoán vị nên:
    \begin{enumerate}[label=\alph*)]
        \item Ta có sáu cách sắp xếp: \{1,x,x,x\}, \{x,1,x,x\}, \{x,x,x,1\}.
        \begin{itemize}
            \item Cách đầu cho ta $3!=6$ cách.
            \item Cách sau cho ta $3!-2!=4$ cách (loại đi các trường hợp \{4,1,x,x\})
            \item Cách cuối chỉ cho ta $2!=2$ cách.
        \end{itemize}
        Vậy có tổng cộng 12 cách, xác suất là 1/2.
        \item Tương tự câu trên nhưng đảo vai trò của 4 và 1, ta cũng có xác suất là 1/2.
        \item Ta có sáu cách sắp xếp: \{4,x,x,x\}, \{x,4,x,x\}.
        \begin{itemize}
            \item Cách đầu cho ta $3!=6$ cách.
            \item Cách sau cho ta $3!-2.2!=2$ cách (loại bỏ trường hợp \{1,4,x,x\} và \{2,4,x,x\})
        \end{itemize}
        Vậy tổng cộng có 8 cách, xác suất là 1/3.
        \item Ta chỉ có 1 cách duy nhất: \{4,x,x,x\} vì vậy có $3!=6$ cách, vậy xác suất là 1/4.
        \item Ta có cách sắp xếp: \{4,x,x,x\}, sẽ có $3!=6$ cách để đặt 3 vào 3 chỗ còn lại, nhưng mỗi cách lại có $2!=2$ cách đặt 1 và 2, ta loại trường hợp 1 đứng trước 2, vậy chỉ còn 1 cách, suy ra tổng cộng có 3 cách, vậy xác suất là 3/24=1/8.
    \end{enumerate}
\end{proof}
\subsection*{Bài 10}
Xác suất của các biến cố sau là bao nhiêu nếu ta ngẫu nhiên chọn một hoán vị của 26 chữ cái viết thường trong bảng chữ cái tiếng Anh?
\begin{enumerate}[label=\alph*)]
    \item 13 chữ cái đầu tiên trong hoán vị tuân theo thứ tự bảng chữ cái.
    \item a là chữ cái đầu tiên và z là chữ cái cuối cùng.
    \item a và z đứng kế nhau trong hoán vị.
    \item a và b không đứng cùng nhau trong hoán vị.
    \item a và z cách nhau ít nhất 23 chữ cái trong hoán vị.
    \item z đứng trước a và b trong hoán vị.
\end{enumerate}
\begin{proof}
    Ta có tổng cộng 26! hoán vị.
    \begin{enumerate}[label=\alph*)]
        \item Ta chọn 13 chữ cái sau cùng ngẫu nhiên nên ta có $P^{13}_{26}$ chỉnh hợp, còn 13 chữ cái còn lại ta chỉ cần xếp theo thứ tự bảng chữ cái (1 cách xếp), vậy xác suất là $\frac{P^{13}_{26}}{26!}\approx1.606.10^{-10}$.
        \item Ta chọn trước a và z, còn lại $24!$ hoán vị, vậy xác suất là $\frac{24!}{26!}=\frac{1}{625}$.
        \item Có 25 trường hợp a và z có thể đứng kế nhau, từ \{a,z,xxx...\} đến \{...xxx,a,z\}, mỗi trường hợp có $24!$ hoán vị nên tổng cộng có $24!.25=25!$ hoán vị, xác suất là $\frac{1}{26}$.
        \item Xem b như z trong trường hợp trên, ta có $25!$ thỏa mãn điều kiện này, và vì đây là biến cố đối nên ta lấy phần bù $1-\frac{1}{26}=\frac{25}{26}$.
        \item Chỉ có 2 trường hợp \{a,...z,x\} và \{x,a,...,z\}, mỗi trường hợp có $24!$ hoán vị nên tổng cộng có $24!.2$ hoán vị, xác suất là $\frac{24!.2}{25!}=\frac{2}{25.26}=\frac{1}{325}$.
        \item Ta sẽ tìm xác suất của biến cố đảo, hay số hoán vị mà a hoặc b hoặc cả hai đứng trước z. Ta có:
        $$\underbrace{\dots a\dots b\dots}_{n-k-1\text{ chữ cái}}z\underbrace{\dots}_{k\text{ kí tự}}$$
        Xét $k$ kí tự sau z, ta có $P^{k}_{24}$ chỉnh hợp (26 chữ cái trừ z, a hoặc b ra), với mỗi chỉnh hợp như vậy ta hoán vị $n-k-1$ kí tự trước z, suy ra số cách có thể là: $$\sum_{k=0}^{24}(P^k_{24}.(25-k)!)$$
        Chia kết quả thu được cho $26!$, ta được xác suất 1/2.
    \end{enumerate}
\end{proof}
\subsection*{Bài 11}
Cho rằng E và F là hai biến cố sao cho $p(E)=0.7$ và $p(F)=0.5$. Chứng minh rằng $p(E\cup F)\geq0.7$ và $p(E\cap F)\geq0.2$.
\begin{proof}
    \par Theo nguyên lí Dirichlet, giả sử $|\Omega|=10$, ta có $|E|=7$ và $|F|=5$, vậy nếu ta chọn ngẫu nhiên 7 kết quả trong $\Omega$ cho E, trong trường hợp xấu nhất sẽ có 3 kết quả của F rơi vào 3 kết quả mà E chưa chọn, còn 2 kết quả chắc chắn rằng E đã chọn. Nếu may mắn hơn, sẽ có một vài kết quả mà cả E và F đều chưa chọn, xem như $|\Omega|<10$, vậy giao của hai tập hợp sẽ lớn hơn, nhưng giá trị nhỏ nhất là 0.2.
    \par Tương tự với $E\cup F$, trong trường hợp tốt nhất F sẽ chọn 3 kết quả mà E chưa chọn, và 2 kết quả khác mà E đã chọn, như vậy hợp của chúng là toàn bộ không gian mẫu, hay $p(E\cup F)=1$, kém may mắn hơn thì sẽ có một vài kết quả mà cả E và F không chọn, vậy kết quả sẽ nhỏ hơn 1, nhưng sẽ luôn không dưới 0.7, kể cả trong trường hợp xấu nhất, toàn bộ kết quả của F đều được chọn bởi E. Lúc đó $p(E\cup F)=\max(p(E),p(F))=0.7$.
\end{proof}
\subsection*{Bài 12}
Cho rằng E và F là hai biến cố sao cho $p(E)=0.8$ và $p(F)=0.6$. Chứng minh rằng $p(E\cup F)\geq0.8$ và $p(E\cap F)\geq0.4$.
\begin{proof}
    Chứng minh tương tự bài trên, vì $|E|$ và $|F|$ đều tăng 1 đơn vị nên $p(E\cap F)$ tăng 0.2, còn $p(E\cup F)=\max(p(E),p(F))=0.8$.
\end{proof}
\subsection*{Bài 13}
Cho rằng E và F là hai biến cố, thì $p(E\cap F)\geq p(E)+p(F)-1$.
\begin{proof}
    Vì E và $F$ chung không gian mẫu nên ta có $p(E\cup F)=1$. Nếu ta coi $p(E)$ như $|E|$, $p(F)$ như $|F|$, dựa theo định lí $|E\cap F|=|E|+|F|-|E\cup F|$, ta suy ra điều phải chứng minh.
\end{proof}
\subsection*{Bài 26}
Cho E là biến cố một dãy bit độ dài 3 có số bit 1 lẻ, F là biến cố dãy bit đó bắt đầu bằng bit 1. E và F có độc lập nhau không?
\begin{proof}
    Ta có $|\Omega|=2^3=8$ khả năng.
    Để tìm |E|, ta tìm số dãy bit không có bit 1 nào (000), và 2 bit 1 (11x, cho ta 3 hoán vị), vậy $|E|=4$.
    Để tìm |F|, dãy bit có dạng 1xx, dùng quy tắc nhân ta thu được 4 dãy bit thỏa, vậy $|F|=4$.
    Tính toán ta được $p(E)=1/2$, $p(F)=1/2$ và $p(E)p(F)=1/4$.
    Để tìm $|E\cap F|$, ta có 4 dãy bit bắt đầu bằng 1, nhưng trừ đi các trường hợp có 1 bit 0 (101, 110), thu được 2 dãy bit thỏa, vậy $p(E\cap F)=1/4$. Kết luận E và F độc lập nhau.
\end{proof}
\subsection*{Bài 31}
Tìm xác suất một gia đình có năm người con không có con trai, nếu các giới tính là độc lập và nếu \begin{enumerate}[label=\alph*)]
    \item một cậu bé và một cô bé có xác suất như nhau.
    \item xác suất của một cậu bé là 0.51.
    \item xác suất của đứa trẻ thứ $i$ là con trai là $0.51-(i/100)$
\end{enumerate}
\begin{proof}
    Ta sử dụng công thức của định lý 2: Xác suất đúng $k$ lần thành công trong $n$ lần thử Bernoulli với tỉ lệ thành công $p$ và tỉ lệ thất bại $q=1-p$ là $C^k_np^kq^{n-k}$. Ta coi việc sinh được con gái là thành công. Từ đề bài ta có $n=k=5$.
    \begin{enumerate}[label=\alph*)]
        \item $p=0.5,q=0.5\rightarrow$ xác suất là $\frac{1}{32}=0.03125$.
        \item $p=0.49,q=0.51\rightarrow$ xác suất là $0.49^5\approx0.02825$.
        \item Vì xác suất thất bại thay đổi theo thứ tự con, từ công thức ta rút ra được, nếu $p^5$ xuất hiện trong công thức, vậy $p$ phải là hằng số. Ta sẽ áp dụng thành: $$C^5_5.\prod_{i=1}^5(1-0.51+\frac{i}{100})=0.03795012$$.
    \end{enumerate}
\end{proof}