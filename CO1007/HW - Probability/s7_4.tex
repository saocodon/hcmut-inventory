\section*{Phần 7.4}
\subsection*{Bài 6}
Tìm giá trị kỳ vọng khi một tờ vé số \$1 được mua bởi người mua, biết người này trúng \$10,000,000 nếu tờ vé số chứa sáu con số được chọn từ tập hợp \{1, 2, 3,\dots, 50\}, ngược lại thì không nhận được gì.
\begin{proof}
    Vì mỗi số trong tập hợp đều có xác suất ra là 1/50, áp dụng định lý 1, ta thu được giá trị kỳ vọng:
    $$E(X)=\frac{1}{50}(1+2+\dots+50)=\frac{1275}{50}=25.5$$
\end{proof}
\subsection*{Bài 7}
Bài thi cuối kì môn cấu trúc rời rạc có 50 câu hỏi đúng/sai, mỗi câu 2 điểm và 25 câu hỏi trắc nghiệm, mỗi câu 4 điểm. Xác suất Linda trả lời đúng một câu hỏi đúng/sai là 0.9 và một câu hỏi trắc nghiệm là 0.8. Điểm của cô ta có thể là bao nhiêu?
\begin{proof}
    Gọi biến ngẫu nhiên $X(i)$ là số điểm đạt được khi trả lời câu hỏi $i$. Áp dụng định lí 1, ta thu được đáp án:
    $E(X)=50*0.9*2+25*0.8*4=170$ điểm.
\end{proof}
\subsection*{Bài 9}
Cho rằng khả năng $x$ xuất hiện trong một dãy gồm $n$ số nguyên khác biệt là 2/3 và xác suất $x$ bằng mỗi phần tử trong dãy là như nhau. Tìm số thao tác so sánh trung bình dùng bởi thuật toán tìm kiếm tuần tự để tìm $x$ hoặc đưa ra đáp án $x$ không có trong dãy.
\begin{proof}
    Nhận xét:
    \begin{itemize}
        \item Ta cần $2i+1$ phép so sánh nếu $x$ bằng $a_i$ trong danh sách, $i$ phép để so sánh 2 phần tử, $i$ phép để kiểm tra $i\leq n$, và 1 phép so sánh $i\leq n$ ngoài vòng lặp (lệnh \texttt{if}).
        \item Ta cần $2n+2$ phép nếu $x$ không có trong danh sách, khi cần $2n+1$ phép để so sánh với toàn bộ phần tử, 1 phép so sánh $i\leq n$ ở cuối vòng lặp.
    \end{itemize}
    Vì xác suất $x$ bằng một phần tử trong dãy là $p=2/3$, suy ra xác suất $x$ tồn tại trong dãy là $\frac{p}{n}$, ta đặt $q=1-2/3=1/3$ là xác suất $x$ khác một phần tử trong dãy, suy ra xác suất $x$ không tồn tại trong dãy là $\frac{q}{n}$. Gọi $E$ là số thao tác kì vọng, ta có:
    \begin{align*}
        E&=\frac{3p}{n}+\frac{5p}{n}+\dots+\frac{(2n+1)p}{n}+(2n+2)q\\
        &=\frac{p}{n}(3+5+\dots+(2n+1))+(2n+2)q\\
        &=\frac{p}{n}[(n+1)^2-1]+(2n+2)q\\
        &=p(n+2)+(2n+2)q
    \end{align*}
    Thay $p$ và $q$ vào biểu thức, ta được $$E=\frac{2}{3}(n+2)+(2n+2)\frac{1}{3}=\frac{4n}{3}+2$$
\end{proof}
\subsection*{Bài 38}
Cho rằng số lon soda pop được bơm mỗi ngày tại một nhà máy đóng chai là một biến ngẫu nhiên với giá trị kỳ vọng 10,000 và phương sai 1000.
\begin{enumerate}[label=\alph*)]
    \item Dùng bất đẳng thức Markov (bài 37) để tìm cận trên của xác suất mà nhà máy có thể bơm hơn 11,000 lon mỗi ngày.
    \item Dùng bất đẳng thức Chebyshev để tìm cận dưới của xác suất mà nhà máy có thể bơm từ 9000 đến 11,000 lon mỗi ngày.
\end{enumerate}
\begin{proof}.
    \begin{enumerate}[label=\alph*)]
        \item Ta có $p(X(s)\geq 11000)\leq\frac{10000}{11000}\approx0.909$.
        \item Ta có $p(|X(s)-10000|\geq 1000)\geq\frac{10000}{1000^2}=\frac{10000}{1000000}=0.01$.
    \end{enumerate}
\end{proof}