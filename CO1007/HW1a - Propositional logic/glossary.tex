\section*{Giải thích từ ngữ}
\begin{itemize}
    \item \textit{true}: là tính đúng đắn của một mệnh đề hoặc biểu thức. Ví dụ, bất đẳng thức $a+b\geq2\sqrt{ab}$ là một mệnh đề đúng, tính đúng của nó đã được chứng minh dựa theo bất đẳng thức Cauchy.
    \item \textit{truth}: là một giá trị logic đại diện cho sự đúng đắn so với thực tế. Trong logic mệnh đề, \textit{truth value} (giá trị đúng) được kí hiệu là T, \textit{false value} (giá trị sai) được kí hiệu là F. Vậy bất đẳng thức Cauchy là một mệnh đề đúng, kí hiệu là T.
    \item \textit{valid}: là một mệnh đề hợp logic. Một mệnh đề logic hợp lệ không nhất thiết phải có cơ sở đúng, nhưng nếu kết luận được dẫn dắt từ cơ sở một cách hợp lệ thì vẫn được coi là một mệnh đề hợp lệ. Ví dụ, "Trái đất quay quanh mặt trời" là một mệnh đề hợp lệ, và tính đúng của nó đã được kiểm chứng thông qua hiểu biết của con người về thiên văn học.
    \item \textit{correct}: là sự đạt yêu cầu, có tính đúng đắn của một luận lí logic (một đoạn mã, một mệnh đề\dots) về một khía cạnh nào đó, ví dụ trong ngôn ngữ, "Mặt Trăng làm từ pho mát" là một mệnh đề đạt yêu cầu về mặt ngôn ngữ, nhưng nó không đạt yêu cầu về mặt logic.
    \item \textit{fallacy}: là một lỗi xảy ra trong chứng minh làm cho một luận điểm hợp lệ. Có hai loại là \textit{lỗi hình thức} và \textit{lỗi phi hình thức}.
    \begin{itemize}
        \item Lỗi hình thức: một lỗi sai lầm trong chứng minh làm cho kết luận không thể chứng minh được. Ví dụ, biến mệnh đề P là "trời mưa", Q là "đường ướt". Vậy ta có mệnh đề $P\rightarrow Q$ là "Nếu trời mưa thì đường ướt". Mệnh đề này đúng, nhưng nếu suy ngược lại ($Q\rightarrow P$) thì không chắc chắn được.
        \item Lỗi phi hình thức: là một bài chứng minh trong đó nó không dựa trên các cơ sở đã được nêu ra. Ví dụ: \begin{itemize}
            \item A nói: "Chúng ta nên đầu tư vào năng lượng xanh để tránh biến đổi khí hậu."
            \item B trả lời: "Bạn chỉ nói vậy vì bạn đang làm cho một công ty năng lượng xanh, tôi không tin bạn."
        \end{itemize}
        Trong ví dụ này, luận điểm "A đầu tư vào năng lượng xanh" không dựa trên cơ sở rằng "A làm cho công ty năng lượng xanh" vì chưa chắc chắn rằng A có đủ tiền, có quyền xây dựng\dots để thực hiện hoá kế hoạch.
    \end{itemize}
    \item \textit{contradiction} là một mệnh đề luôn sai. Ví dụ mệnh đề A: "$1+1=3$".
    \item \textit{paradox}: là một mệnh đề có cơ sở đúng nhưng dẫn tới kết luận mâu thuẫn hoặc sai. Ví dụ như nghịch lí nói dối: \begin{itemize}
        \item Mệnh đề $P$: "Câu này sai".
        \item Nếu $P$ đúng, thì "Câu này sai" là đúng ($P$ sai).
        \item Nếu $P$ sai, thì "Câu này sai" là sai ($P$ đúng).
    \end{itemize}
    Vậy ta thấy nghịch lí bị mâu thuẫn (phủ định 2 lần).
    \item \textit{counterexample}: là một trường hợp cho thấy một mệnh đề là sai. Ví dụ, một giáo viên đã nói: "Cái lớp này rất lười", nhưng chưa hẳn là cả lớp đều lười, có thể có một số học sinh giỏi rất siêng năng, vậy các bạn học sinh giỏi này là \textit{counterexample} của trường hợp trên.
    \item \textit{premise}: là một tập hợp các mệnh đề tạo ra cơ sở cho một kết luận. Ví dụ trong bất đẳng thức Cauchy, nó được sinh ra từ mệnh đề cơ sở: bình phương của một số luôn không âm.
    \item \textit{assumption} (giả sử): là một mệnh đề được thừa nhận là đúng mà không chứng minh để làm cơ sở cho một kết luận. Một giả sử có thể sai. Ví dụ, ta giả sử "Ngày hôm đó là ngày trong tuần, tất cả mọi người đều đi làm", ta sẽ chứng minh một cái gì đó dựa trên giả thiết này, tuy vẫn sẽ có một số lượng người không đi làm nhưng số lượng đó là rất ít nếu ta xét trên không gian mẫu đủ lớn. Hoặc giả sử này có thể sai, nếu ta xét ở một đất nước nào đó, ngày nghỉ là một ngày nào đó trong tuần.
    \item \textit{presumption}: là một tư tưởng được thừa nhận là đúng dựa trên các cơ sở hợp lí, được sử dụng trong trường hợp có một số bằng chứng nhưng vẫn không thể xác định được kết quả. Ví dụ, ta cho rằng bệnh A gồm có một số triệu chứng nào đó, nhưng các triệu chứng đó chỉ \textbf{có thể} là bệnh A, không thể chắc chắn rằng đó là bệnh A được.
    \item \textit{axiom}: là một số quy tắc được thừa nhận là đúng mà không chứng minh, được dùng để vận hành một hệ thống trong khoa học máy tính. Ví dụ như phép giao hoán, phép kết hợp trong các phép toán cơ bản với số thực.
    \item \textit{hypothesis} (giả thiết): là một mệnh đề là một sự giải thích, hoặc đoán và có thể kiểm tra được. Nó có thể là giải thích cách một thuật toán hoạt động, hoặc một hàm số nào đó diễn tả hoạt động của một thuật toán. Chúng ta sẽ dùng các công cụ kiểm tra để làm cho giả thiết này sai. Ví dụ, thuật toán Bubble Sort có độ phức tạp là $O(n^2)$.
    \item \textit{conjecture}: là một tư tưởng được tin là đúng dựa theo quan sát hoặc trải nghiệm, nhưng chưa được chứng minh. Nó được dùng làm điểm xuất phát cho việc nghiên cứu hoặc khám phá sâu hơn. Ví dụ như ta biết rằng, không tồn tại ba số nguyên dương $a,b,c$ nào sao cho $a^n+b^n=c^n$ với $n>2$ (Định lý cuối cùng của Fermat), trừ Fermat ra chưa ai có thể chứng minh được định lý này, nhưng bài chứng minh của ông đã không được ghi lại.
    \item \textit{tautology}: là một mệnh đề luôn đúng, dù mệnh đề nguyên tử của nó có đúng hay sai. Ví dụ, cho mệnh đề $P$, ta có mệnh đề $P\lor\neg P$ (P đúng hoặc sai), ta có mệnh đề này luôn đúng, bất kể $P$ đúng hay sai.
    \item \textit{satisfiable} (có thể thoả mãn): một công thức được coi là có thể thoả mãn khi có một bộ giá trị boolean (T hoặc F) được gán vào các biến của nó để làm cho cả công thức đúng. Ví dụ, ta có mệnh đề $(P\lor Q)\land(\neg P\lor R)$ đúng nếu $P$ đúng, $Q$ sai, $R$ đúng.
    \item \textit{contingency}: là một mệnh đề có thể đúng hoặc sai, dựa vào các giá trị được gán cho biến mệnh đề. Ví dụ $P\rightarrow Q$ có thể đúng hoặc sai, tuỳ giá trị của $P$ và $Q$.
    \item \textit{inference} (suy luận): là quá trình sinh ra những mệnh đề khác dựa vào những mệnh đề đã có bằng định lý. Ví dụ, nếu $a=b$ thì $a^2=b^2$.
    \item \textit{argument} là một tập hợp các mệnh đề. Mệnh đề đầu tiên gọi là cơ sở (premise), mệnh đề cuối cùng gọi là kết luận, ví dụ trong chứng minh bất đẳng thức Cauchy, mệnh đề cơ sở là $(a+b)^2\geq0$, kết luận là $a+b\geq2\sqrt{ab}$.
    \item \textit{reasoning} (biện luận): là toàn bộ quá trình sinh ra những mệnh đề khác dựa trên định lý hoặc suy luận. Quá trình suy luận chính là một trong nhiều bước của quá trình biện luận. Một bài chứng minh cũng chính là một bài biện luận.
    \item \textit{variable} (biến mệnh đề): là các giá trị đầu vào có thể đúng hoặc sai, chúng tạo thành mệnh đề, và chúng cũng có thể là một mệnh đề riêng, thường \textbf{được kí hiệu} bằng các chữ cái như P, Q, R, S\dots trong logic mệnh đề. Ví dụ, "Trời mưa" là một biến mệnh đề (ví dụ trong câu "Trời mưa thì tôi sẽ mang ô").
    \item \textit{arity}: nói đến số tham số hoặc hạng tử trong một biểu thức. Ví dụ: \begin{itemize}
        \item Hàm rỗng (null\textbf{ary}): không có tham số. Ví dụ: $f()=2$.
        \item Hàm một ngôi (un\textbf{ary}): có một tham số. Ví dụ: $f(x)=2x$.
        \item Hàm hai ngôi (bi\textbf{nary}): có hai tham số. Ví dụ: $f(x,y)=2xy$.
        \item Hàm ba ngôi (terti\textbf{ary}): có ba tham số. Ví dụ: $f(x,y,z)=2xyz$.
        \item Hàm $n$ ngôi ($n$-\textbf{ary}): có $n$ tham số. Ví dụ: $$f(x_1,x_2,\dots,x_n)=2\prod_{i=1}^n x_i$$
    \end{itemize}
\end{itemize}